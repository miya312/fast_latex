%このファイルではプリアンブル部分をまとめて書いている
%texworks -stylesheet .\texlive\2022\bin\win32\user.cssは、texの文字だか背景だかを変化させるファイル
\usepackage[dvipdfmx]{graphicx,xcolor}

\usepackage[export]{adjustbox} % includegraphicsなどでmax/min width/heightを使えるようにする
% dvipdfmxは日本語のときのみかく
\usepackage{amsmath, amssymb, amsfonts, mathtools} % おまじない

\usepackage[top=20truemm,bottom=20truemm,left=25truemm,right=20truemm, includeheadfoot, dvipdfmx]{geometry} % 余白の設定 includeheadfootでヘッダー、フッターを含めて空白を開ける 恐らくfancyhdrよりも前にする必要がある

\usepackage{braket, derivative} % braket, 微分
\usepackage{physics2} % physicsの代わり
	\usephysicsmodule{ab, doubleprod, diagmat, xmat}
	\renewcommand{\Re}{\operatorname{Re}}
	\renewcommand{\Im}{\operatorname{Im}}
	\newcommand{\Tr}{\operatorname{Tr}}
	\newcommand{\rank}{\operatorname{rank}}
	\newcommand{\order}{\mathcal{O}}

% \usepackage{mathcomp} % 単位 physicsを使う場合はmathcomp
\usepackage{siunitx} % 単位 physicsを使う場合はqtyが衝突するので使わない
\usepackage[version=4]{mhchem} % 化学式

\setcounter{tocdepth}{2} % 目次の深さ
\usepackage[dvipdfmx]{hyperref} % リンク付きref
\usepackage{pxjahyper} % 日本語refの文字化け防止 (u)pLaTexのみ
	\hypersetup{%
		%  dvipdfmx,
		setpagesize=false,%
		bookmarks=true,%
		bookmarksdepth=tocdepth,%
		bookmarksnumbered=true,%
		%  allcolors=blue,%
		hidelinks,
		pdftitle={タイトル},
		pdfauthor={著者の名前},
	}

\usepackage{nameref} % 章の名前を参照
% \usepackage{wtref} % refコマンドを増やす
	% \newref{sec, eq}
	% \newref[scope=chapter]{fig, tb}
	% \setrefstyle{eq}{refcmd=(\ref{#1})}
	% \setrefstyle{sec}{prefix=第, suffx=章}
\usepackage{zref-xr} % ファイルをまたいだref。ref系のパッケージの最後に読み込む
	\zxrsetup{toltxlabel}

\usepackage{ascmac} % 枠
\usepackage{bm} % 太字

\usepackage{here} % 図表の位置
\usepackage{enumitem} % リスト
\usepackage{multirow} % 表で複数マスにまたがるもの
% \usepackage{tikz} % tikz pgfplotsを読み込むと自動で読み込まれるっぽい
% \usetikzlibrary{calc}
\usepackage{pgfplotstable, pgfplots} % 表, グラフ
	\pgfplotsset{
		compat=1.18,
		table/col sep=comma
	}
	\usetikzlibrary{calc}
\usepackage{csvsimple} % csv読み込み
\usepackage[]{caption} % キャプション
\usepackage[subrefformat=parens]{subcaption} % subキャプション
	\captionsetup{compatibility=false}

% \usepackage[compress=false]{graphicscache} % 画像のキャッシュ(動いてない)
\usepackage{fancyhdr} % ヘッダー
	\pagestyle{fancy}
	\lhead{\leftmark} %ヘッダ左
	\chead{} %ヘッダ中央
	\rhead{\rightmark} %ヘッダ右.コンパイルした日付を表示
	\renewcommand{\chaptermark}[1]{\markboth{第\ \normalfont\thechapter\ 章~#1}{}}
	%\renewcommand{\sectionmark}[1]{\markright{\thesection #1}{}}
	\lfoot{} %フッタ左
	\cfoot{\thepage} %フッタ中央.ページ番号を表示
	\rfoot{} %フッタ右
	% \renewcommand{\headrulewidth}{} %ヘッダの罫線
	% \renewcommand{\footrulewidth}{} %フッタの罫線

\usepackage{environ, ifthen, xparse, comment} % 環境の定義, if文, コマンド定義, コメントアウト
% \usepackage{dirtree} % ディレクトリツリー

\usepackage[deluxe]{otf} % utf入力 pxchfonの設定によっては必須
\usepackage[ipaex]{pxchfon} % フォント埋め込み(文字化け防止)
% \usepackage[english, japanese]{babel}
\usepackage[style=phys, articletitle=false, biblabel=brackets, chaptertitle=false, pageranges=false]{biblatex} % bib
	% https://qiita.com/shiro_takeda/items/fac1351495f32c224a28
	% \renewbibmacro{in:}{} % in: Some journal の "in:" を取る
	% https://paper3510mm.github.io/latex/biblatex.html
	\addbibresource{../bib/refs.bib}
	\ExecuteBibliographyOptions{
		sortcites=true,
		% backref=true,
		date=year
	}

	% 日本語文章への対応
	% 下記参考資料ではlangidにenglishなどjapanese以外がある場合はエラーを吐くため、japaneseのみに反応するようにしてenglishがあっても大丈夫なようにしている。
	% http://granular.blog39.fc2.com/blog-entry-76.html
	% https://tex.stackexchange.com/questions/498682/disabling-the-printing-of-language-only-while-using-the-macro-based-on-language
	\newbibmacro*{finalnamedelim:japanese}{\multinamedelim}

	\renewcommand*{\finalnamedelim}{%
		\iffieldequalstr{langid}{japanese}
		{\usebibmacro*{finalnamedelim:\strfield{langid}}}
		{\ifnumgreater{\value{liststop}}{2}{\finalandcomma}{}%
			\addspace\bibstring{and}\space}}

	\newbibmacro*{name:given-family:japanese}[4]{%
		\usebibmacro{name:delim}{#1#2}%
		\usebibmacro{name:hook}{#1#2}%
		#1\bibnamedelimc#2}

	\DeclareNameFormat{given-family}{%
		\iffieldequalstr{langid}{japanese}
		{\usebibmacro*{name:given-family:\strfield{langid}}
			{\namepartfamily}
			{\namepartgiven}
			{\namepartprefix}
			{\namepartsuffix}}%
		{\ifgiveninits
			{\usebibmacro{name:given-family}
				{\namepartfamily}
				{\namepartgiveni}
				{\namepartprefix}
				{\namepartsuffix}}
			{\usebibmacro{name:given-family}
				{\namepartfamily}
				{\namepartgiven}
				{\namepartprefix}
				{\namepartsuffix}}}
		\usebibmacro{name:andothers}}


% 以下コマンド定義 -----------------------------------------------------------------
\NewDocumentCommand{\TODO}{m}{{\Large \textcolor{red}{!!TODO!! #1}}}
\newcommand{\intii}{\int_{-\infty}^{\infty}}
\newcommand{\intzi}{\int_{0}^{\infty}}
\newcommand{\intiz}{\int_{-\infty}^{0}}

\newcommand{\ctext}[1]{\raise0.2ex\hbox{\textcircled{\scriptsize{#1}}}} % https://livingdead0812.hatenablog.com/entry/20161005/1475654232

\makeatletter
\renewcommand{\chapter}{% https://qiita.com/hermite2053/items/d869f8673838080a238b
	\if@openleft\cleardoublepage\else
	\if@openright\cleardoublepage\else\clearpage\fi\fi
	%\plainifnotempty %元: \thispagestyle{plain}
	\global\@topnum\z@
	\if@english \@afterindentfalse \else \@afterindenttrue \fi
	\secdef
	{\@omit@numberfalse\@chapter}%
	{\@omit@numbertrue\@schapter}}
\makeatother

% csvをプロットする
% #1 [] : 表示場所, default H (Hはhtbpと一緒に使わない!)
% #2 {} : キャプション
% #3 {} : axisのオプション: xlabel, ylabelなど
% #4 {} : addplotのオプション: x index, y indexなど
% #5 {} : csvのパス
% #6 [] : ラベル, default #2
% #7 [] : ラベルの fig: の部分, default fig:
\NewDocumentCommand{\plotcsv}{O{H} m m m m o o}{
	\begin{figure}[#1]
		\centering
		\begin{tikzpicture}
			\begin{axis}[
				height=13\baselineskip,
				width=0.8\columnwidth,
				#3
			]
				\addplot table [only marks,#4] {#5};
			\end{axis}
		\end{tikzpicture}
		\caption{#2}\label{\IfValueTF{#7}{#7}{fig:}\IfValueTF{#6}{#6}{#2}}
	\end{figure}
}

% figure -> includegraphics
% #1 [] : 表示場所, default H (Hはhtbpと一緒に使わない!)
% #2 {} : キャプション
% #3 [] : includegraphics のオプション, keepaspectratioは常に有効, default height=10\baselineskip
% #4 {} : 画像のパス
% #5 [] : ラベル, default #2
% #6 [] : ラベルの fig: の部分, default fig:
\NewDocumentCommand{\fig}{O{H} m O{max height=15\baselineskip} m o o}{
	\begin{figure}[#1]
		\centering
		\includegraphics[keepaspectratio, max width=0.9\columnwidth, #3]{#4}
		\caption{#2}\label{\IfValueTF{#6}{#6}{fig:}\IfValueTF{#5}{#5}{#2}}
	\end{figure}
}

\NewDocumentCommand{\figtwo}{O{H} m o m m o o}{
	\begin{figure}[#1]
		\centering
		\begin{minipage}[b]{0.45\columnwidth}
			\centering
			\includegraphics[keepaspectratio, max width=1\columnwidth, \IfValueT{#3}{#3}]{#4}
		\end{minipage}
		% \hspace*{0.05\columnwidth}
		\begin{minipage}[b]{0.45\columnwidth}
			\centering
			\includegraphics[keepaspectratio, max width=1\columnwidth, \IfValueT{#3}{#3}]{#5}
		\end{minipage}
		\caption{#2}\label{\IfValueTF{#7}{#7}{fig:}\IfValueTF{#6}{#6}{#2}}
	\end{figure}
}

\makeatletter
% セクションなど章のタイトル+label
% #1 {} : 章のネスト深さ。0 = chapter, 1 = section, 2 = subsection, 3 = subsubsection
% #2 {} : 章のタイトル
% #3 [] : タイトルにコマンドなどが入っている場合のラベル用テキスト
\NewDocumentCommand{\sct}{m +m o}{
	\ifthenelse{\equal{#1}{0}}{
		% chapter
		\IfValueTF{#3}{
			\chapter{#2}\label{chap:#3}
			\def\chaptername{#3}
		}{
			\chapter{#2}\label{chap:#2}
			\def\chaptername{#2}
		}
	}{
	\ifthenelse{\equal{#1}{1}}{
		% section
		\IfValueTF{#3}{
			\section{#2}\label{sec:\chaptername:#3}
			\def\sectionname{#3}
		}{
			\section{#2}\label{sec:\chaptername:#2}
			\def\sectionname{#2}
		}
	}{
	\ifthenelse{\equal{#1}{2}}{
		% subsection
		\IfValueTF{#3}{
			\subsection{#2}\label{s_sec:\chaptername:\sectionname:#3}
			\def\subsectionname{#3}
		}{
			\subsection{#2}\label{s_sec:\chaptername:\sectionname:#2}
			\def\subsectionname{#2}
		}
	}{
	\ifthenelse{\equal{#1}{3}}{
		% subsubsection
		\IfValueTF{#3}{
			\subsubsection{#2}\label{ss_sec:\chaptername:\sectionname:\subsectionname:#3}
			\def\subsubsectionname{#3}
		}{
			\subsubsection{#2}\label{ss_sec:\chaptername:\sectionname:\subsectionname:#2}
			\def\subsubsectionname{#2}
		}
	}{
		% error
		\textcolor{red}{\Huge エラー : \textbackslash sctの第1引数は0から3の整数です @ #2}
	}
	}
	}
	}
}
\makeatother